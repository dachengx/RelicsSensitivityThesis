% !TeX encoding = UTF-8
% !TeX program = xelatex
% !TeX spellcheck = en_US

\documentclass[degree=bachelor]{thuthesis}
  % 学位 degree:
  %   doctor | master | bachelor | postdoc
  % 学位类型 degree-type:
  %   academic(默认)| professional
  % 语言 language
  %   chinese(默认)| english
  % 字体库 fontset
  %   windows | mac | fandol | ubuntu
  % 建议终版使用 Windows 平台的字体编译


% 论文基本配置,加载宏包等全局配置
% !TeX root = ./thuthesis-example.tex

% 论文基本信息配置

\thusetup{
  %******************************
  % 注意:
  %   1. 配置里面不要出现空行
  %   2. 不需要的配置信息可以删除
  %   3. 建议先阅读文档中所有关于选项的说明
  %******************************
  %
  % 输出格式
  %   选择打印版(print)或用于提交的电子版(electronic),前者会插入空白页以便直接双面打印
  %
  output = print,
  %
  % 标题
  %   可使用“\\”命令手动控制换行
  %
  title  = {LXeTPC 和 HPGe 探测反应堆中微子-原子核相干弹性散射的实验灵敏度研究},
  title* = {An Introduction to \LaTeX{} Thesis Template of Tsinghua
            University v\version},
  %
  % 学位
  %   1. 学术型
  %      - 中文
  %        需注明所属的学科门类,例如:
  %        哲学、经济学、法学、教育学、文学、历史学、理学、工学、农学、医学、
  %        军事学、管理学、艺术学
  %      - 英文
  %        博士:Doctor of Philosophy
  %        硕士:
  %          哲学、文学、历史学、法学、教育学、艺术学门类,公共管理学科
  %          填写“Master of Arts“,其它填写“Master of Science”
  %   2. 专业型
  %      直接填写专业学位的名称,例如:
  %      教育博士、工程硕士等
  %      Doctor of Education, Master of Engineering
  %   3. 本科生不需要填写
  %
  degree-name  = {工学硕士},
  degree-name* = {Master of Science},
  %
  % 培养单位
  %   填写所属院系的全名
  %
  department = {工程物理系},
  %
  % 学科
  %   1. 学术型学位
  %      获得一级学科授权的学科填写一级学科名称,其他填写二级学科名称
  %   2. 工程硕士
  %      工程领域名称
  %   3. 其他专业型学位
  %      不填写此项
  %   4. 本科生填写专业名称,第二学位论文需标注“(第二学位)”
  %
  discipline  = {工程物理},
  discipline* = {Computer Science and Technology},
  %
  % 姓名
  %
  author  = {徐大成},
  author* = {Xue Ruini},
  %
  % 指导教师
  %   中文姓名和职称之间以英文逗号“,”分开,下同
  %
  supervisor  = {杨丽桃, 助理教授},
  supervisor* = {Professor Zheng Weimin},
  %
  % 副指导教师
  %
  associate-supervisor  = {高飞, 助理教授},
  associate-supervisor* = {Professor Chen Wenguang},
  %
  % 联合指导教师
  %
  % co-supervisor  = {某某某, 教授},
  % co-supervisor* = {Professor Mou Moumou},
  %
  % 日期
  %   使用 ISO 格式;默认为当前时间
  %
  % date = {2019-07-07},
  %
  % 是否在中文封面后的空白页生成书脊(默认 false)
  %
  include-spine = false,
  %
  % 密级和年限
  %   秘密, 机密, 绝密
  %
  % secret-level = {秘密},
  % secret-year  = {10},
  %
  % 博士后专有部分
  %
  % clc                = {分类号},
  % udc                = {UDC},
  % id                 = {编号},
  % discipline-level-1 = {计算机科学与技术},  % 流动站(一级学科)名称
  % discipline-level-2 = {系统结构},          % 专业(二级学科)名称
  % start-date         = {2011-07-01},        % 研究工作起始时间
}

% 载入所需的宏包

% 定理类环境宏包
\usepackage{amsthm}
% 也可以使用 ntheorem
% \usepackage[amsmath,thmmarks,hyperref]{ntheorem}

\thusetup{
  %
  % 数学字体
  % math-style = GB,  % GB | ISO | TeX
  math-font  = xits,  % sitx | xits | libertinus
}

% 可以使用 nomencl 生成符号和缩略语说明
% \usepackage{nomencl}
% \makenomenclature

% 表格加脚注
\usepackage{threeparttable}

% 表格中支持跨行
\usepackage{multirow}

% 固定宽度的表格。
% \usepackage{tabularx}

% 跨页表格
\usepackage{longtable}

% 算法
\usepackage{algorithm}
\usepackage{algorithmic}

% 量和单位
\usepackage{siunitx}

% 参考文献使用 BibTeX + natbib 宏包
% 顺序编码制
\usepackage[sort]{natbib}
\bibliographystyle{thuthesis-numeric}

% 著者-出版年制
% \usepackage{natbib}
% \bibliographystyle{thuthesis-author-year}

% 本科生参考文献的著录格式
% \usepackage[sort]{natbib}
% \bibliographystyle{thuthesis-bachelor}

% 参考文献使用 BibLaTeX 宏包
% \usepackage[style=thuthesis-numeric]{biblatex}
% \usepackage[style=thuthesis-author-year]{biblatex}
% \usepackage[style=apa]{biblatex}
% \usepackage[style=mla-new]{biblatex}
% 声明 BibLaTeX 的数据库
% \addbibresource{ref/refs.bib}

% 定义所有的图片文件在 figures 子目录下
\graphicspath{{figures/}}

% 数学命令
\makeatletter
\newcommand\dif{%  % 微分符号
  \mathop{}\!%
  \ifthu@math@style@TeX
    d%
  \else
    \mathrm{d}%
  \fi
}
\makeatother

% hyperref 宏包在最后调用
\usepackage{hyperref}



\begin{document}

% 封面
\maketitle

% 学位论文指导小组、公开评阅人和答辩委员会名单
% 本科生不需要
% \input{data/committee}

% 使用授权的说明
\copyrightpage
% 将签字扫描后授权文件 scan-copyright.pdf 替换原始页面
% \copyrightpage[file=scan-copyright.pdf]

\frontmatter
% !TeX root = ../thuthesis-example.tex

% 中英文摘要和关键字

\begin{abstract}
  中微子与原子核的相干弹性散射(CE$\nu$NS)是一种标准模型下的粒子物理过程。
  本文对以核反应堆作为中微子源、使用液氙时间投影室探测器和高纯锗探测器测量该过程的灵敏度进行了预测。
  通过设计建造一个极低阈值、极低本底并可在地面附近运行的液氙时间投影室。
  在信号所在能区,电子反冲本底预计可以压低至0.1$\left(\si{kg}\cdot\si{day}\cdot\si{keV}\right)^{-1}$水平,
  $\mu$子引起的核反冲本底可以压低至0.5$\left(\si{kg}\cdot\si{day}\right)^{-1}$水平。
  在一台堆芯热功率250$\si{GW}$商业发电反应堆附近运行一年,30千克量级的液氙时间投影室探测器预计探测到超过3000个CE$\nu$NS信号。
  通过设计建造一台本底水平与CDEX-10类似的5千克高纯锗探测器,在同等条件下,预计探测到超过1000个CE$\nu$NS信号。
  利用探测器到的CE$\nu$NS过程,两种探测器对中微子超标准模型有效相互作用和低动量转移下的弱混合角的测量将在类似实验中达到领先水平。

  % 关键词用“英文逗号”分隔,输出时会自动处理为正确的分隔符
  \thusetup{
    keywords = {中微子与原子核的相干弹性散射, 灵敏度预测, 本底模拟, 超标准模型有效相互作用},
  }
\end{abstract}

\begin{abstract*}
  An abstract of a dissertation is a summary and extraction of research work and contributions.
  Included in an abstract should be description of research topic and research objective, brief introduction to methodology and research process, and summary of conclusion and contributions of the research.
  An abstract should be characterized by independence and clarity and carry identical information with the dissertation.
  It should be such that the general idea and major contributions of the dissertation are conveyed without reading the dissertation.

  An abstract should be concise and to the point.
  It is a misunderstanding to make an abstract an outline of the dissertation and words “the first chapter”, “the second chapter” and the like should be avoided in the abstract.

  Keywords are terms used in a dissertation for indexing, reflecting core information of the dissertation.
  An abstract may contain a maximum of 5 keywords, with semi-colons used in between to separate one another.

  % Use comma as separator when inputting
  \thusetup{
    keywords* = {keyword 1, keyword 2, keyword 3, keyword 4, keyword 5},
  }
\end{abstract*}


% 目录
\tableofcontents

% 符号对照表
% !TeX root = ../thuthesis-example.tex

\begin{denotation}[3cm]
  \item[$E_\nu$] 反应堆中微子能量
  \item[$\epsilon$] 事件在探测器中沉积能量
  \item[$T_\mathrm{nr}$] 核反冲能量
  \item[$T_\mathrm{er}$] 电子反冲能量
  \item[$G_F$] 费米常数
  \item[$\theta_w$] Weinberg 角
  \item[$M$] 原子核质量
  \item[$N_\gamma$] 液氙中沉积能量产生光子个数
  \item[$N_e$] 液氙中沉积能量产生漂移电子个数
  \item[$L_y$] 液氙光产额
  \item[$Q_y$] 液氙电子产额
  \item[$N_\mathrm{hit}$] PMT阵列接收到的击中数目
  \item[$Q_F$] 淬灭因子
  \item[$E_\mu$] $\mu$子能量
\end{denotation}



% 也可以使用 nomencl 宏包,需要在导言区
% \usepackage{nomencl}
% \makenomenclature

% 在这里输出符号说明
% \printnomenclature[3cm]

% 在正文中的任意为都可以标题
% \nomenclature{PI}{聚酰亚胺}
% \nomenclature{MPI}{聚酰亚胺模型化合物,N-苯基邻苯酰亚胺}
% \nomenclature{PBI}{聚苯并咪唑}
% \nomenclature{MPBI}{聚苯并咪唑模型化合物,N-苯基苯并咪唑}
% \nomenclature{PY}{聚吡咙}
% \nomenclature{PMDA-BDA}{均苯四酸二酐与联苯四胺合成的聚吡咙薄膜}
% \nomenclature{MPY}{聚吡咙模型化合物}
% \nomenclature{As-PPT}{聚苯基不对称三嗪}
% \nomenclature{MAsPPT}{聚苯基不对称三嗪单模型化合物,3,5,6-三苯基-1,2,4-三嗪}
% \nomenclature{DMAsPPT}{聚苯基不对称三嗪双模型化合物(水解实验模型化合物)}
% \nomenclature{S-PPT}{聚苯基对称三嗪}
% \nomenclature{MSPPT}{聚苯基对称三嗪模型化合物,2,4,6-三苯基-1,3,5-三嗪}
% \nomenclature{PPQ}{聚苯基喹噁啉}
% \nomenclature{MPPQ}{聚苯基喹噁啉模型化合物,3,4-二苯基苯并二嗪}
% \nomenclature{HMPI}{聚酰亚胺模型化合物的质子化产物}
% \nomenclature{HMPY}{聚吡咙模型化合物的质子化产物}
% \nomenclature{HMPBI}{聚苯并咪唑模型化合物的质子化产物}
% \nomenclature{HMAsPPT}{聚苯基不对称三嗪模型化合物的质子化产物}
% \nomenclature{HMSPPT}{聚苯基对称三嗪模型化合物的质子化产物}
% \nomenclature{HMPPQ}{聚苯基喹噁啉模型化合物的质子化产物}
% \nomenclature{PDT}{热分解温度}
% \nomenclature{HPLC}{高效液相色谱(High Performance Liquid Chromatography)}
% \nomenclature{HPCE}{高效毛细管电泳色谱(High Performance Capillary lectrophoresis)}
% \nomenclature{LC-MS}{液相色谱-质谱联用(Liquid chromatography-Mass Spectrum)}
% \nomenclature{TIC}{总离子浓度(Total Ion Content)}
% \nomenclature{\textit{ab initio}}{基于第一原理的量子化学计算方法,常称从头算法}
% \nomenclature{DFT}{密度泛函理论(Density Functional Theory)}
% \nomenclature{$E_a$}{化学反应的活化能(Activation Energy)}
% \nomenclature{ZPE}{零点振动能(Zero Vibration Energy)}
% \nomenclature{PES}{势能面(Potential Energy Surface)}
% \nomenclature{TS}{过渡态(Transition State)}
% \nomenclature{TST}{过渡态理论(Transition State Theory)}
% \nomenclature{$\increment G^\neq$}{活化自由能(Activation Free Energy)}
% \nomenclature{$\kappa$}{传输系数(Transmission Coefficient)}
% \nomenclature{IRC}{内禀反应坐标(Intrinsic Reaction Coordinates)}
% \nomenclature{$\nu_i$}{虚频(Imaginary Frequency)}
% \nomenclature{ONIOM}{分层算法(Our own N-layered Integrated molecular Orbital and molecular Mechanics)}
% \nomenclature{SCF}{自洽场(Self-Consistent Field)}
% \nomenclature{SCRF}{自洽反应场(Self-Consistent Reaction Field)}



% 正文部分
\mainmatter
\input{data/chap01}
\input{data/chap02}
\input{data/chap03}
\input{data/chap04}


% 其他部分
\backmatter

% 插图和附表清单
% 本科生的插图索引和表格索引需要移至正文之后、参考文献前
% \listoffiguresandtables  % 插图和附表清单(仅限研究生)
\listoffigures           % 插图清单
\listoftables            % 附表清单

% 参考文献
\bibliography{ref/refs}  % 参考文献使用 BibTeX 编译
% \printbibliography       % 参考文献使用 BibLaTeX 编译

% 致谢
% !TeX root = ../thuthesis-example.tex

\begin{acknowledgements}
  衷心感谢导师物理系高飞教授和工程物理系杨丽桃教授对本人的精心指导。他们的言传身教将使我终生受益。

  感谢中国科学技术大学林箐教授,中山大学魏月环教授和肖翔教授在研究中对本人的指导和建议。

  感谢工程物理系续本达教授将我引入粒子物理领域的科研启蒙。

  感谢在物理系实习期间,刘可欣、蔡畅、谢凌峰、贾成杰等同学们的热情讨论与经验分享。

  感谢XENON合作组同事们,石申阳、袁澜清、徐子豪、马越等在工作中的相互帮助。
\end{acknowledgements}


% 声明
\statement
% 将签字扫描后的声明文件 scan-statement.pdf 替换原始页面
% \statement[file=scan-statement.pdf]
% 本科生编译生成的声明页默认不加页脚,插入扫描版时再补上;
% 研究生编译生成时有页眉页脚,插入扫描版时不再重复。
% 也可以手动控制是否加页眉页脚
% \statement[page-style=empty]
% \statement[file=scan-statement.pdf, page-style=plain]

% 附录
% 本科生需要将附录放到声明之后,个人简历之前
\appendix
% \input{data/appendix-survey}       % 本科生:外文资料的调研阅读报告
% % !TeX root = ../thuthesis-example.tex

\begin{translation}
\label{cha:translation}

\title{中微子与原子核的相干弹性散射的观测}
\maketitle

\tableofcontents

\begin{abstract}
  尽管其预测的截面是低能中微子相互作用中最大的,四十年来我们未探测到中微子与原子核的相干弹性散射。
  这种相互作用为我们提供了研究中微子相互作用的新机会,以及探测器小型化和其潜在技术应用的可能性。
  我们用低本底的14.6$\si{kg}$掺钠碘化铯晶体,以6.7$\sigma$的置信度观测到了橡树岭国际实验室散裂中子源中微子的这种过程。
  在高信噪比的条件下,我们观测到了标准模型预言中这种过程对应的能量和时间特性。
  借助原始数据,我们也提升了对中微子超标准模型夸克相互作用的限制。
\end{abstract}

中微子常见的特性是与其他物质相互作用的概率极低,这使得他们星际穿梭的过程中也几乎不损失能量。
所以,对他们的探测一般需要借助巨大的靶物质量(几吨到几万吨)。
对中微子弱中性流相互作用的发现预示着中微子可以通过与夸克(quark)交换Z玻色子来相互作用。
不久以后人们发现这种相互作用会导致中微子与原子核中的所有核子发生相干性的作用。
这种效应只有在中微子与原子核交换的动量小于原子核的的大小,等效地,限制了这种效应只发生于几十$\si{MeV}$的中微子。
相比于与单个核子相互作用,相干作用对相互作用截面的增强效果是非常大的,截面大约正比于原子核中中子个数的平方。对于非常重的原子核和足够强的中微子源,相互作用的增强会使得需要的探测器质量急剧减小,降至几千克水平。

在首次理论预言之后的四十三年中,实验上一直未能探测到中微子-原子核相干弹性散射(CE$\nu$NS)。
这从某种程度上说是令人惊讶的,尤其是考虑到其截面远超其他已经被检验的相互作用,以及可用的中微子源:太阳、大气和地球,超新星爆发,核反应堆,散裂源,和特定的放射源。长期缺乏实验证据的主要原因是,作为相互作用的唯一产物,低能(几个$\si{keV}$)的核反冲难以探测。相比于相同能量的最小电离粒子,反冲的核子在探测器材料中产生可测量的闪烁光或电离的能力非常弱。更重的介质原子核会使得CE$\nu$NS截面增大,但是同时也使得最大的反冲能减小,这种权衡也加重了探测的难度。

对CE$\nu$NS探测的兴趣不只来自于补全标准模型预言的中微子相互作用的拼图。
自CE$\nu$NS被预言之后,其也被认为是拓展我们对中微子性质认知的工具。这方面的研究包括惰性中微子、中微子磁矩、新粒子引入的超标准模型相互作用,原子核结构,以及提升对弱核力的限制。
不仅如此,中微子探测器质量的降低可能会促成一系列技术应用,比如非侵入性的核反应堆监测。
CE$\nu$NS也被认为主导了中子星和核坍缩中的中微子输运过程。对当前最受关注的暗物质候选者:弱相互作用大质量粒子(WIMPs)的探测,依赖于相同的、未受检验的WIMP-原子核相互作用的相干性增强,同时也很快将被无法消除的太阳和大气中微子CE$\nu$NS限制。
这种相互作用的重要性已经引发了许多潜在的CE$\nu$NS探测器提案:超导器件、量能器、改进的半导体探测器、液态惰性元素,以及无机半导体等等。

橡树岭国家实验室的散裂中子源通过加速器加速高能质子(约1$\si{GeV}$)撞击水银靶,可以产生世界上最强的中子束流。这些束流服务于中子散射仪以及从事许多交叉学科的用户。
散裂源也会同时产生可观数量的中微子,这些中微子在静止的$\pi$介子——质子相互作用的副产品——的衰变过程中产生。这样得到的低能中微子对CE$\nu$NS探测非常有利。
三种不同味的中微子(瞬发$\mu$子中微子$\nu_\mu$,缓发电子中微子$\nu_\mathrm{e}$,缓发$\mu$子反中微子$\bar{\nu_\mu}$)被产生出来,他们各自有特定的能量和时间分布,并且在能量相同的时候CE$\nu$NS截面相近。
在束流产生时,大约每天产生$5\times10^{20}$个靶上质子(protons-on-target, POT),每个质子对每种味产生平均约0.08个各向同性放出的中微子。
中微子以脉冲形式发射,频率为$60\si{Hz}$,每次产生POT的时间约为$1\si{\mu s}$。这对于探测是很有利的:使得我们可以把稳定地影响CE$\nu$NS探测器的环境本底,与在POT触发后约$10\si{\mu s}$内发生的中微子产生的信号区别开。
在触发前类似长度的时间窗可以用于测量非时变本底的性质和频率,之后可以进一步从信号中扣除本底的贡献。
所有设施会一直共用散裂中子源产生的$60\si{Hz}$的触发信号。

% 书面翻译的参考文献
\bibliographystyle{unsrtnat}
\bibliography{ref/appendix}

% 书面翻译对应的原文索引
\begin{translation-index}
  \nocite{akimov_observation_2017}
  \bibliographystyle{unsrtnat}
  \bibliography{ref/appendix}
\end{translation-index}

\end{translation}
  % 本科生:外文资料的书面翻译
\input{data/appendix}

% 个人简历、在学期间完成的相关学术成果
% 本科生可以附个人简历,也可以不附个人简历
% \input{data/resume}

% 指导教师/指导小组学术评语
% 本科生不需要
% \input{data/comments}

% 答辩委员会决议书
% 本科生不需要
% \input{data/resolution}

% 本科生的综合论文训练记录表(扫描版)
\record{file=scan-record.pdf}

\end{document}
