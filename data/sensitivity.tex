% !TeX root = ../main.tex

\chapter{LXeTPC与HPGe探测器对CE$\nu$NS探测的灵敏度}
\label{sec:sensitivity}

有了信号模型和本底估计后我们可以进行利用CE$\nu$NS限制某些物理参数的灵敏度计算。
信号和本底在能谱或$\mathrm{S2}$谱中服从泊松分布(Poisson distribution),当特定的一个或几个物理参数改变后,
一定测量曝光量下,对信号(或本底)的预期将会随之改变;若此时测量结果,在考虑了统计涨落及系统误差等因素时仍然无法与模型符合,
我们会宣布发现了超出(excess)或反常(anomaly);另一方面,若测量结果比较符合信号,我们也可以给出对某些物理量测量的置信区间。
本章建立灵敏度分析框架,假定信号与本底相符合,分析CE$\nu$NS测量对中微子相互作用相关的参数的限制能力,即给出置信区间。

\section{中微子超标准模型有效相互作用测量}

\subsection{S2-Only分析}

\subsection{S1-S2分析}

\section{低动量转移下的弱混合角测量}

\section{低能核反冲的液氙光电产额测量}
