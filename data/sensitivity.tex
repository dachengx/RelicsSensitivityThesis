% !TeX root = ../main.tex

\chapter{LXeTPC与HPGe探测器对CE$\nu$NS探测的灵敏度}
\label{sec:sensitivity}

有了信号模型和本底估计后我们可以进行利用CE$\nu$NS限制某些物理参数的灵敏度计算。
信号和本底在能谱或$\mathrm{S2}$谱中服从泊松分布(Poisson distribution),当特定的一个或几个物理参数改变后,
一定测量曝光量下,对信号(或本底)的预期将会随之改变;若此时测量结果,在考虑了统计涨落及系统误差等因素时仍然无法与模型符合,
我们会宣布发现了超出(excess)或反常(anomaly);另一方面,若测量结果比较符合信号,我们也可以给出对某些物理量测量的置信区间。
本章建立灵敏度分析框架,假定信号与本底相符合,分析CE$\nu$NS测量对中微子相互作用相关的参数的限制能力,即给出置信区间。

\section{中微子超标准模型有效相互作用测量}

中微子超标准模型相互作用(non-standard-interaction, NSI)是一种模型无关的(model-independent)弱相互作用相关的截面参数化\cite{barranco_probing_2005}。
与电子反中微子CE$\nu$NS相关的两个参数为$\epsilon_{ee}^{uV},\epsilon_{ee}^{dV}$,见式\ref{eq:nsi}(公式中使用了偶偶核简化)。

\begin{align}
    \label{eq:nsi}
    \frac{\mathrm{d}\sigma(E_\nu)}{\mathrm{d}T} &= \frac{G_F^2 M}{2\pi}G_V^2\left[1+(1-\frac{T}{E_{\nu}})^2-\frac{MT}{E_{\nu}^2}\right] \\
    G_V &= ((g_V^p+2\epsilon_{ee}^{uV}+\epsilon_{ee}^{dV})Z+(g_V^n+\epsilon_{ee}^{uV}+2\epsilon_{ee}^{dV})N)F^V(Q^2)
\end{align}

标准模型中这两个参数为0,公式退化为\ref{eq:cevns_even_even}。
下面分别分析LXeTPC上S2-Only分析和S1-S2分析。以及HPGe探测器上分析的实验灵敏度。假设曝光量为$30\si{kg\cdot y}$。

\subsection{S2-Only分析}

S2-Only分析仅使用$\mathrm{S2}$作为分析维度。$\epsilon_{ee}^{uV}=\epsilon_{ee}^{dV}=0$时,
信号$\mathrm{S2}$分布如图\ref{fig:S2_rate}。

假设4个及以上的漂移电子探测效率为100\%,,选取$\mathrm{S2}$范围为$[80, 200]\si{PE}$,分为3个bin。
在该$\mathrm{S2}$范围中相应曝光量下CE$\nu$NS的总期望个数为4490.6个,核反冲本底的总期望个数为897.3,
电子反冲本底的总期望个数为34.1。

测量的主要系统误差来源分为三部分:反应堆中微子总流强不确定度$\sigma_{\gamma_f}$、本底不确定度$\beta_i$、液氙核反冲量子产额不确定度$t_i$。
高能反应堆中微子能谱的理论计算、反应堆功率监测结果、裂变份额计算均会引入一定的不确定度\cite{an_improved_2017},
我们保守地设置$\sigma_{\gamma_f}=10\%$,且与中微子能量无关。

核素活度测量结果不确定度,以及模拟中不同几何和辐射过程模型的使用,均会导致核反冲本底和电子反冲本底的不确定度,
我们简单地设置核反冲本底不确定度$\sigma_{\beta_1}=10\%$,电子反冲本底不确定度$\sigma_{\beta_2}=10\%$,且与本底谱形状无关。

液氙核反冲量子产额相对不确定度$t_1$和$t_2$分别对应核反冲中光和电子产额,
本文采用2020年XENON1T的测量结果\cite{aprile_search_2021},并向低能延拓,如图\ref{fig:lxe_nr_yield}。
当$t_1=\pm1$时,$L_y=\langle L_y\rangle\pm\sigma_{L_y}$,$t_2=\pm1$时,$Q_y=\langle Q_y\rangle\pm\sigma_{Q_y}$。
液氙核反冲量子产额的不确定度将是最重要的系统误差,$t_2$取$-1,0,1$时,$\mathrm{S2}$谱的变化如图\ref{fig:xe_rate_prediction_t2}。

\begin{figure}
  \centering
  \includesvg[width=0.7\linewidth]{figures/xe_rate_prediction_t2.svg}
  \caption{\label{fig:xe_rate_prediction_t2} S2-Only分析中的$\mathrm{S2}$谱,$\langle Q_y\rangle$为XENON1T测量的期望。}
\end{figure}

表\ref{tab:sys_error}总结了各种系统误差来源与取值,适用于本文所有灵敏度预测。

\begin{table}
  \centering
  \caption{各种系统误差来源与取值}
  \begin{tabular}{ccc}
    \toprule
    名称 & 符号 & 取值 \\
    \midrule
    反应堆中微子流强不确定度 & $\sigma_{\gamma_f}$ & $10\%$ \\
    核反冲本底不确定度 & $\sigma_{\beta_1}$ & $10\%$ \\
    电子反冲本底不确定度 & $\sigma_{\beta_2}$ & $10\%$ \\
    核反冲光产额相对不确定度 & $t_1$ & 图\ref{fig:lxe_nr_yield} \\
    核反冲电子产额相对不确定度 & $t_2$ & 图\ref{fig:lxe_nr_yield} \\
    \bottomrule
  \end{tabular}
  \label{tab:sys_error}
\end{table}

建立$\chi^2$,如式\ref{eq:nsi_ll}。

\begin{align}
    \label{eq:nsi_ll}
    \chi^2 &= \sum_j\left[\frac{\left(N_{obs,j} - N_{exp,j}(\epsilon^{uV}_{ee}, \epsilon^{dV}_{ee}, t_2)(1 + \gamma_f) 
    - \sum_{i}^2 B_{i,j}(1 + \beta_i)\right)^2}{\sigma^2_{stat,j}}\right] 
    + t_2^2 + \sum_{i}^2(\frac{\beta_i}{\sigma^2_{\beta_i}})^2 + (\frac{\gamma_f}{\sigma^2_{\gamma_f}})^2
\end{align}

其中$j$遍历各$\mathrm{S2}$bins,$B_{i,j}$为本底在不同$\mathrm{S2}$bins上的期望,$\beta_i$为不确定度向期望的传递。
取$\Delta chi^2=4.61$以确定二维参数上的90\%置信区间(confidential interval),如图\ref{fig:nsi_sensitivity_s2only}。

\begin{figure}
  \centering
  \includesvg[width=0.7\linewidth]{figures/nsi_sensitivity_s2only.svg}
  \caption{\label{fig:nsi_sensitivity_s2only} 一次S2-Only测量对中微子超标准模型有效参数的限制(绿色阴影)。
  同时列出了COHERENT的首次结果作为对比\cite{akimov_observation_2017}。}
\end{figure}

\subsection{S1-S2分析}

\begin{figure}
  \centering
  \includesvg[width=0.7\linewidth]{figures/nsi_sensitivity_s1s2.svg}
  \caption{\label{fig:nsi_sensitivity_s1s2} 一次S1-S2测量对中微子超标准模型有效参数的限制(绿色阴影)。
  同时列出了COHERENT的首次结果作为对比\cite{akimov_observation_2017}。}
\end{figure}

\section{低动量转移下的弱混合角测量}

\section{低能核反冲的液氙光电产额测量}
