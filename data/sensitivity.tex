% !TeX root = ../main.tex

\chapter{LXeTPC与HPGe探测器对CE$\nu$NS探测的灵敏度}
\label{sec:sensitivity}

有了信号模型和本底估计后我们可以进行利用CE$\nu$NS限制某些物理参数的灵敏度计算。
信号和本底在能谱或$\mathrm{S2}$谱中服从泊松分布(Poisson distribution),当特定的一个或几个物理参数改变后,
一定测量曝光量下,对信号(或本底)的预期将会随之改变;若此时测量结果,在考虑了统计涨落及系统误差等因素时仍然无法与模型符合,
我们会宣布发现了超出(excess)或反常(anomaly);另一方面,若测量结果比较符合信号,我们也可以给出对某些物理量测量的置信区间。
本章建立灵敏度分析框架,假定信号与本底相符合,分析CE$\nu$NS测量对中微子相互作用相关的参数的限制能力,即给出置信区间。

\section{中微子超标准模型有效相互作用测量}

中微子超标准模型相互作用(non-standard-interaction, NSI)是一种模型无关的(model-independent)弱相互作用相关的截面参数化\cite{barranco_probing_2005}。
与电子反中微子CE$\nu$NS相关的两个参数为$\epsilon_{ee}^{uV},\epsilon_{ee}^{dV}$,见式\ref{eq:nsi}(公式中使用了偶偶核简化)。

\begin{align}
    \label{eq:nsi}
    \frac{\mathrm{d}\sigma(E_\nu)}{\mathrm{d}T} &= \frac{G_F^2 M}{2\pi}G_V^2\left[1+(1-\frac{T}{E_{\nu}})^2-\frac{MT}{E_{\nu}^2}\right] \\
    G_V &= ((g_V^p+2\epsilon_{ee}^{uV}+\epsilon_{ee}^{dV})Z+(g_V^n+\epsilon_{ee}^{uV}+2\epsilon_{ee}^{dV})N)F^V(Q^2)
\end{align}

标准模型中这两个参数为0,公式退化为\ref{eq:cevns_even_even}。
下面分别分析LXeTPC上S2-Only分析和S1-S2分析。以及HPGe探测器上分析的实验灵敏度。

\subsection{S2-Only分析}

S2-Only分析仅使用$\mathrm{S2}$作为分析维度。$\epsilon_{ee}^{uV}=\epsilon_{ee}^{dV}=0$时,
信号$\mathrm{S2}$分布如图\ref{fig:S2_rate}。



\subsection{S1-S2分析}

\section{低动量转移下的弱混合角测量}

\section{低能核反冲的液氙光电产额测量}
