% !TeX root = ../thuthesis-example.tex

\begin{translation}
\label{cha:translation}

\title{中微子与原子核的相干弹性散射的观测}
\maketitle

\tableofcontents

\begin{abstract}
  尽管其预测的截面是低能中微子相互作用中最大的,四十年来我们未探测到中微子与原子核的相干弹性散射。
  这种相互作用为我们提供了研究中微子相互作用的新机会,以及探测器小型化和其潜在技术应用的可能性。
  我们用低本底的14.6$\si{kg}$掺钠碘化铯晶体,以6.7$\sigma$的置信度观测到了橡树岭国际实验室散裂中子源中微子的这种过程。
  在高信噪比的条件下,我们观测到了标准模型预言中这种过程对应的能量和时间特性。
  借助原始数据,我们也提升了对中微子超标准模型夸克相互作用的限制。
\end{abstract}

中微子常见的特性是与其他物质相互作用的概率极低,这使得他们星际穿梭的过程中也几乎不损失能量。
所以,对他们的探测一般需要借助巨大的靶物质量(几吨到几万吨)。

% 书面翻译的参考文献
\bibliographystyle{unsrtnat}
\bibliography{ref/appendix}

% 书面翻译对应的原文索引
\begin{translation-index}
  \nocite{akimov_observation_2017}
  \bibliographystyle{unsrtnat}
  \bibliography{ref/appendix}
\end{translation-index}

\end{translation}
