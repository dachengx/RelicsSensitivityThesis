% !TeX root = ../thuthesis-example.tex

\begin{translation}
\label{cha:translation}

\title{中微子与原子核的相干弹性散射的观测}
\maketitle

\tableofcontents

\begin{abstract}
  尽管其预测的截面是低能中微子相互作用中最大的,四十年来我们未探测到中微子与原子核的相干弹性散射。
  这种相互作用为我们提供了研究中微子相互作用的新机会,以及探测器小型化和其潜在技术应用的可能性。
  我们用低本底的14.6$\si{kg}$掺钠碘化铯晶体,以6.7$\sigma$的置信度观测到了橡树岭国际实验室散裂中子源中微子的这种过程。
  在高信噪比的条件下,我们观测到了标准模型预言中这种过程对应的能量和时间特性。
  借助原始数据,我们也提升了对中微子超标准模型夸克相互作用的限制。
\end{abstract}

中微子常见的特性是与其他物质相互作用的概率极低,这使得他们星际穿梭的过程中也几乎不损失能量。
所以,对他们的探测一般需要借助巨大的靶物质量(几吨到几万吨)。
对中微子弱中性流相互作用的发现预示着中微子可以通过与夸克(quark)交换Z玻色子来相互作用。
不久以后人们发现这种相互作用会导致中微子与原子核中的所有核子发生相干性的作用。
这种效应只有在中微子与原子核交换的动量小于原子核的的大小,等效地,限制了这种效应只发生于几十$\si{MeV}$的中微子。
相比于与单个核子相互作用,相干作用对相互作用截面的增强效果是非常大的,截面大约正比于原子核中中子个数的平方。
对于非常重的原子核和足够强的中微子源,相互作用的增强会使得需要的探测器质量急剧减小,降至几千克水平。

在首次理论预言之后的四十三年中,实验上一直未能探测到中微子-原子核相干弹性散射(CE$\nu$NS)。
这从某种程度上说是令人惊讶的,尤其是考虑到其截面远超其他已经被检验的相互作用,以及可用的中微子源:太阳、大气和地球,超新星爆发,核反应堆,散裂源,和特定的放射源。
长期缺乏实验证据的主要原因是,作为相互作用的唯一产物,低能(几个$\si{keV}$)的核反冲难以探测。
相比于相同能量的最小电离粒子,反冲的核子在探测器材料中产生可测量的闪烁光或电离的能力非常弱。
更重的介质原子核会使得CE$\nu$NS截面增大,但是同时也使得最大的反冲能减小,这种权衡也加重了探测的难度。

对CE$\nu$NS探测的兴趣不只来自于补全标准模型预言的中微子相互作用的拼图。
自CE$\nu$NS被预言之后,其也被认为是拓展我们对中微子性质认知的工具。这方面的研究包括惰性中微子、中微子磁矩、新粒子引入的超标准模型相互作用,
原子核结构,以及提升对弱核力的限制。不仅如此,中微子探测器质量的降低可能会促成一系列技术应用,比如非侵入性的核反应堆监测。
CE$\nu$NS也被认为主导了中子星和核坍缩中的中微子输运过程。对当前最受关注的暗物质候选者:弱相互作用大质量粒子(WIMPs)的探测,
依赖于相同的、未受检验的WIMP-原子核相互作用的相干性增强,同时也很快将被无法消除的太阳和大气中微子CE$\nu$NS限制。
这种相互作用的重要性已经引发了许多潜在的CE$\nu$NS探测器提案:超导器件、量能器、改进的半导体探测器、液态惰性元素,以及无机半导体等等。

橡树岭国家实验室的散裂中子源通过加速器加速高能质子(约1$\si{GeV}$)撞击水银靶,可以产生世界上最强的中子束流。这些束流服务于中子散射仪以及从事许多交叉学科的用户。
散裂源也会同时产生可观数量的中微子,这些中微子在静止的$\pi$介子——质子相互作用的副产品——的衰变过程中产生。这样得到的低能中微子对CE$\nu$NS探测非常有利。
三种不同味的中微子(瞬发$\mu$子中微子$\nu_\mu$,缓发电子中微子$\nu_\mathrm{e}$,缓发$\mu$子反中微子$\bar{\nu_\mu}$)被产生出来,
他们各自有特定的能量和时间分布,并且在能量相同的时候CE$\nu$NS截面相近。
在束流产生时,大约每天产生$5\times10^{20}$个靶上质子(protons-on-target, POT),每个质子对每种味产生平均约0.08个各向同性放出的中微子。
中微子以脉冲形式发射,频率为$60\si{Hz}$,每次产生POT的时间约为$1\mu\si{s}$。这对于探测是很有利的:使得我们可以把稳定地影响CE$\nu$NS探测器的环境本底,
与在POT触发后约$10\mu\si{s}$内发生的中微子产生的信号区别开。
在触发前类似长度的时间窗可以用于测量非时变本底的性质和频率,之后可以进一步从信号中扣除本底的贡献。
所有设施会一直共用散裂中子源产生的$60\si{Hz}$的触发信号。

大量中微子释放出的同时,数量同样巨大的瞬发中子将从包围在水银靶的钢铁屏蔽体逃离,淹没在散裂源仪器舱内的CE$\nu$NS探测器。
中子产生的核反冲将远超过中微子引起的核反冲,致使实验不可行。这要求我们研究散裂源设施中的瞬发中子。一条地下走廊,被称为``中微子小巷'',是合适的探测器放置位置。
其到散裂源靶组件有额外的12米厚的无空隙的中子慢化材料(混凝土、砾石)屏蔽。8米等效水(meter of water equivalent,m.w.e.)厚的埋深也提供了对宇宙线的额外屏蔽。
掺钠的碘化铯($\mathrm{CsI[Na]}$)探测器将被放在走廊中最靠近散裂源靶的位置,我们之后会讨论到。

文献中以一个2千克的原型机为例,描述了作为CE$\nu$NS探测介质,掺钠的碘化铯的优势、应用特性和本底研究。
较重的碘和铯核提供了较大的截面,且对CE$\nu$NS几乎一样的响应,同时对低至几个$\si{keV}$的核反冲也能产生足以探测到的闪烁光。我们在最终的14.6$\si{kg}$的$\mathrm{CsI[Na]}$安装到散裂源设施中之前对其进行了额外的标定,并研究了其在相关能区的核反冲响应。
不仅如此,我们还在选定的探测器放置位置进行了专门的前导实验来测量能够到达探测器位置的很小的瞬发中子流强,以及对中微子致中子(neutrino-induced neutron, NIN)可能的最大贡献进行了限制。测量结果表明,CE$\nu$NS信号将远超出束流相关的本底。
相比于探测器原型机,最终得到的晶体探测器的稳定的环境本底在期望上较小,这主要是因为数据分析的细微改进,以及增加的屏蔽体。文献提供了实验设置的更多内容。

图片展示了我们从累计共十四个月时间测量中得到的主要的结果。通过对比在POT触发前$\mathrm{CsI[Na]}$中的信号与POT触发后紧跟着的信号,我们在第二个数据集中的能谱和信号时间分布中,以很高的置信度观察到了超出。

% 书面翻译的参考文献
\bibliographystyle{unsrtnat}
\bibliography{ref/appendix}

% 书面翻译对应的原文索引
\begin{translation-index}
  \nocite{akimov_observation_2017}
  \bibliographystyle{unsrtnat}
  \bibliography{ref/appendix}
\end{translation-index}

\end{translation}
