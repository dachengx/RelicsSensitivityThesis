% !TeX root = ../main.tex

\chapter{总结与展望}

本文建立了LXeTPC测量反应堆中微子CE$\mu$NS信号的本底模拟框架和灵敏度预测框架。

若可以控制地面附近的本底,尤其是$\mu$子引起的核反冲本底,
$30\si{kg}$灵敏体积的LXeTPC对中微子超标准模型有效相互作用参数的灵敏度将达到世界领先,
对若混合角的测量也将填补实验空白。

研究中建立的模拟框架可以用于优化探测器几何设计和本底控制。

研究中对信号相应进行了一定量的近似,如TPC光收集效率和$\mathrm{S2}$增益的计算中未考虑探测器几何效应,
对4个及以上的漂移电子采用了100\%的探测效率。同时未完全考虑探测器的几何细节如聚四氟乙烯反射板、整形环等。
在未来的研究中将建立完整的探测器几何模型进行光学模拟和电场模拟以确定几何效应。

HPGe探测器灵敏度计算中未考虑地面中子本底的水平,应针对$\mu$子设计相应的屏蔽体结构,建立类似的模拟框架。

本底计算中使用了简单的长期平衡假设,在探测器实际运行中将实际测量材料的俄各种放射性含量后进行模拟。
模拟后使用的能量沉积聚类算法的计算效率有限,$\mu$子模拟个数不足够高以消除模拟中的统计误差。

$\mu$子反符合探测器将引入曝光量损失,$\mu$子沉积能量引起的偶然符合等本底也会引入曝光量损失,这也是未来分析中需要考虑的。
