% !TeX root = ../main.tex

\chapter{地面运行的LXeTPC本底研究}
\label{sec:backgrounds}

地面附近运行的LXeTPC将会经受原生宇宙线和次生宇宙线的轰击。
来自外部空间的宇宙线成分以高能质子为主,在穿越地球大气层的过程中与大气分子发生簇射反应,产生大量次级粒子,
主要包括质子、$\mu$子、电子、中子、$\gamma$光子等。宇宙线粒子通量与大气层深度有关,如图\ref{fig:vertical_flux}。

\begin{figure}
    \centering
    \includesvg[width=0.7\linewidth]{figures/vertical_flux.svg}
    \caption{\label{fig:vertical_flux} 宇宙线中不同粒子通量与大气深度的关系\cite{olive_review_2016}。
    $\mu$子和中微子是地面附近通量最高的两种粒子,中微子较少与物质发生相互作用,$\mu$子成为实验中的主要本底。}
\end{figure}

$\mu$子是带电粒子,穿透能力较强。$\mu$子穿越物质的过程中通过电磁相互作用减速,产生电离,
高能($10\si{GeV}$以上)$\mu$子在物质中的能损$-\mathrm{d}E/\mathrm{d}x$大约为$2\si{MeV\cdot cm^{-1}}$。
考虑相对论时间膨胀效应,$\mu$子在物质中能够穿越较长距离。$\mu$子还有可能被原子核俘获,
使靶原子核序数减1,同时放出1个或多个中子。这些中子将是探测器中核反冲本底的主要来源。

利用$\mu$子持续产生电离的特点,可以在探测器外围部署$\mu$子反符合探测器,
当$\mu$子同时在反符合探测器和主探测器产生信号且符合一定条件时时,通过硬件触发或软件筛选,
可以将主探测器中事件标记为$\mu$子事件以区分信号的本底。但物理信号如CE$\nu$NS也有一定几率发生在$\mu$子穿越探测器时,
这时对$\mu$子事件的标记会使物理信号的探测效率降低,等效为曝光量(exposure)的损失,这种效应必须考虑到统计推断中,
否则将高估探测器中产生的信号。

地面附近的$\mu$子分布可以用Gaisser公式或Shukla公式描述。T.K.Gaisser在忽略地球曲率的情况下,
给出了$\mu$子通量随$\mu$子能量$E_\mu$和天顶角(zenith angle)$\theta$的分布。天顶角为入射粒子与地面法线间的夹角\cite{gaisser_cosmic_2016}。

\begin{align}
    \label{eq:gaisser}
    \frac{\mathrm{d}N_\mu}{\mathrm{d}E_\mu\mathrm{d}\Omega} &\approx 
    1400E_\mu^{-2.7}/\left(\si{m^2\cdot s\cdot GeV\cdot sr}\right)\left(\frac{1}{1+\frac{1.1E_\mu\cos\theta}{\epsilon_\pi}}+\frac{0.054}{1+\frac{1.1E_\mu\cos\theta}{\epsilon_\kappa}}\right)
\end{align}

其中$\epsilon_\pi\approx115\si{GeV},\epsilon_\kappa\approx850\si{GeV}$。
Gaisser公式只在高能区间($E_\mu>100/\cos\theta\si{GeV}$)适用。

到达地面附近的$\mu$子包含了$\mu^-$和$\mu^+$。

\section{$\mu$子本底}

\subsection{核反冲本底}

\subsection{电子反冲本底}

\section{材料本底}

\subsection{光电倍增管本底}

\subsection{屏蔽体材料本底}

\subsection{液氙中惰性元素本底}

\section{本底事件的筛选与压低}
