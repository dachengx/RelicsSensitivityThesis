% !TeX root = ../main.tex

\chapter{地面运行的LXeTPC本底研究}
\label{sec:backgrounds}

为了定量地描述实验对某些物理参数的灵敏度,我们需要对实验本底进行估计。
本章使用蒙特卡洛方法(Monte Carlo method),估计两类实验中的主要本底:$\mu$子本底和材料放射性本底,
以及信号筛选条件对两类本底的压低能力。

\section{模拟框架}

Geant4 是辐射探测和粒子物理领域常用的以C++为基础的蒙特卡洛模拟软件包\cite{agostinelli_geant4simulation_2003,allison_geant4_2006,allison_recent_2016}。
PandaX合作组基于Geant4开发了BambooMC软件包用于PandaX实验中的探测器模拟\cite{chen_bamboomc_2021}。
BambooMC具有较强的可拓展性,本文中使用基于BambooMC开发的软件包RelicsSim对地表附近探测器$\mu$事件和材料本底进行模拟。
所用几何如图\ref{fig:relics_g4}所示,结构遵循\ref{fig:relics_geo}的设计。

\begin{figure}
  \begin{subfigure}{.5\textwidth}
    \centering
    \includegraphics[width=1.0\linewidth]{figures/relics_outer.png}
    \caption{\label{fig:relics_outer} LXeTPC屏蔽体几何}
  \end{subfigure}
  \begin{subfigure}{.5\textwidth}
    \centering
    \includegraphics[width=1.0\linewidth]{figures/relics_inner.png}
    \caption{\label{fig:relics_inner} LXeTPC中心探测器几何}
  \end{subfigure}
  \caption{\label{fig:relics_g4} RelicsSim中屏蔽体和探测器几何探测器,绿色体积为液氙4$\pi$反符合探测器,蓝色体积为中心主探测器。}
\end{figure}

探测器基本参数列在表\ref{tab:relics_material}中。

\begin{table}
  \centering
  \caption{LXeTPC几何参数}
  \begin{tabular}{cccc}
    \toprule
    名称 & 几何体 & 参数 & 材料 \\
    \midrule
    外聚乙烯屏蔽体 & 空心长方体 & $216\times216\times226\si{cm^3}$ & 高密度聚乙烯 \\
    铅屏蔽体 & 空心长方体 & $156\times156\times166\si{cm^3}$ & 铅 \\
    内聚乙烯屏蔽体 & 空心长方体 & $126\times126\times136\si{cm^3}$ & 高密度聚乙烯 \\
    铜屏蔽体 & 空心长方体 & $66\times66\times76\si{cm^3}$ & 铅 \\
    空气层 & 空心长方体 & $60\times60\times70\si{cm^3}$ & 空气 \\
    外不锈钢罐体 & 空心圆柱体 & 厚度$0.5\si{cm}$,高$59\si{cm}$,直径$48\si{cm}$ & 不锈钢 \\
    真空隔热层 & 空心圆柱体 & 厚度$5\si{cm}$,高$58\si{cm}$,直径$47\si{cm}$ & 真空 \\
    内不锈钢罐体 & 空心圆柱体 & 厚度$0.5\si{cm}$,高$48\si{cm}$,直径$37\si{cm}$ & 不锈钢 \\
    外层气态氙 & 圆柱体 & 高$5\si{cm}$,直径$36\si{cm}$ & 气态氙 \\
    液氙反符合层 & 空心圆柱体 & 厚度$4\si{cm}$,高$42\si{cm}$,直径$36\si{cm}$ & 液态氙 \\
    内层气态氙 & 圆柱体 & 高$6\si{cm}$,直径$28\si{cm}$ & 气态氙 \\
    液氙主探测器 & 圆柱体 & 高$28\si{cm}$,直径$28\si{cm}$ & 液态氙 \\
    \bottomrule
  \end{tabular}
  \label{tab:relics_material}
\end{table}

液氙主探测器上下分别有64个1英寸光电倍增管:日本滨松R8520,共128个。光电倍增管的几何如图\ref{fig:pmt_geo}。

\begin{figure}
  \begin{subfigure}{.5\textwidth}
    \centering
    \includegraphics[width=1.0\linewidth]{figures/pmt_geo.png}
    \caption{\label{fig:pmt_g4} R8520在Geant4中的几何}
  \end{subfigure}
  \begin{subfigure}{.5\textwidth}
    \centering
    \includesvg[width=1.0\linewidth]{figures/topPMTs.svg}
    \caption{\label{fig:pmt_layout} PMT在探测器$(x,y)$平面上的排布}
  \end{subfigure}
  \caption{\label{fig:pmt_geo} 光电倍增管R8520的几何和在探测器$(x,y)$平面上的排布。
  蓝色透明几何为石英窗;橙色几何为光阴极(photocathode);灰色部分是套管(casing),材质为SAE 304不锈钢。
  排布保证了轴向对称性,可能对未来的位置重建有利。}
\end{figure}

PMT石英窗上富集的${}^{40}\mathrm{K}$和${}^{137}\mathrm{Cs}$可能成为主要的本底来源,具体内容将在第\ref{sec:pmt_background}章中讨论。

\section{本底事件的筛选与压低}

本底事件经常与信号有不同的性质以及测量结果。
我们可以通过选取某些具有区分能力(discrimination power)的性质,在其值域中定义选取信号并进行物理分析的区间,
在尽可能少地损失信号的条件下,去除尽可能多的本底。

对于$\mu$子本底,其可能强烈地在液氙反符合层中沉积能量,可以用反符合层对$\mu$子的标记将其本底压低。
考虑到一个物理事件的时间窗约为$200\mu s$,$\mu$子和中子将可能有足够多的时间在主探测器中多次沉积能量,
而中微子几乎不可能与探测器发生两次相互作用,
所以我们可以通过筛选并去除一个事件窗内有多次能量沉积的事件,对这类本底进行压低。
最后,若一个主探测器中的核反冲事件同时伴随着一个主探测器中的电子反冲事件,则这个核反冲事件有较大可能是$\mu$子引起的而不是中微子。

材料的$\beta$和$\gamma$放射性在中心探测器边缘单位体积沉积的能量比探测器中心多,所以一定程度地舍弃某些本底过高的区域,
对物理信号的搜索是有利的,选取的本底较低的区域定义为灵敏体积。
同时材料本底,有一定概率产生多次散射,如$\gamma$的康普顿散射事件,也可以通过判断是否有多次散射来去除这类本底。

所有与能量相关的筛选条件(或阈值)均需要通过详尽严格的信号模拟来确定;
灵敏体积的选取和设置也需要综合考虑物理信号和本底的空间分布。本文仅根据经验,设置粗略的信号筛选条件,
考察在这些条件下信号和本底的事例率以及相应的实验灵敏度。
针对核反冲本底和电子反冲本底,信号筛选条件列于表\ref{tab:cuts},只有满足表中所有条件的事件才被纳入物理分析。

\begin{table}
  \centering
  \caption{针对核反冲和电子反冲本底的信号筛选条件}
  \begin{tabular}{cc}
    \toprule
    名称 & 条件 \\
    \midrule
    液氙反符合 & 反符合层中最大的电子(核)反冲不大于$100\si{keV}$($500\si{keV}$) \\
    电子反冲单次散射 & 灵敏体积中第二大的散射能量不大于最大电子反冲能量的5\% \\
    核反冲单次散射 & 灵敏体积中第二大的散射能量不大于$1\si{keV}$ \\
    核反冲的电子反冲标记 & 灵敏体积中与核反冲事件同时发生的电子反冲不大于$10\si{keV}$ \\
    灵敏体积 & 气液交界面以下$0.5\si{cm}$到$24.5\si{cm}$,半径$12\si{cm}$的圆柱 \\
    \bottomrule
  \end{tabular}
  \label{tab:cuts}
\end{table}

反符合层中没有漂移电场,只有光信号,我们将在反符合层中设置PMT来探测这他们。
由于核反冲信号的淬灭效应,相同能量的核反冲和电子反冲产生的光子数不同,取淬灭因子(quenching factor)为0.2,
则将反符合层的核反冲阈值设置为电子反冲的$1/0.2=5$倍。

灵敏体积为高$24\si{cm}$,半径为$12\si{cm}$的圆柱。在S2-Only分析中,不使用$\mathrm{S1}$,
信号将没有电子漂移距离$z$,但我们让然可以通过$\mathrm{S2}$波形的事件分布一定程度上判断事件发生的纵向位置,
所以这里对S2-Only和S1-S2分析设置了相同的灵敏体积。

\section{$\mu$子本底}

地面附近运行的LXeTPC将会经受原生宇宙线和次生宇宙线的轰击。
来自外部空间的宇宙线成分以高能质子为主,在穿越地球大气层的过程中与大气分子发生簇射反应,产生大量次级粒子,
主要包括质子、$\mu$子、电子、中子、$\gamma$光子等。宇宙线粒子通量与大气层深度有关,如图\ref{fig:vertical_flux}。

\begin{figure}
    \centering
    \includesvg[width=0.7\linewidth]{figures/vertical_flux.svg}
    \caption{\label{fig:vertical_flux} 宇宙线中不同粒子通量与大气深度的关系\cite{olive_review_2016}。
    $\mu$子和中微子是地面附近通量最高的两种粒子,中微子较少与物质发生相互作用,$\mu$子成为实验中的主要本底。}
\end{figure}

$\mu$子是带电粒子,穿透能力较强。$\mu$子穿越物质的过程中通过电磁相互作用减速,产生电离,
高能($10\si{GeV}$以上)$\mu$子在物质中的能损$-\mathrm{d}E/\mathrm{d}x$大约为$2\si{MeV\cdot cm^{-1}}$。
考虑相对论时间膨胀效应,$\mu$子在物质中能够穿越较长距离。$\mu$子还有可能被原子核俘获,
使靶原子核序数减1,同时放出1个或多个中子。这些中子将是探测器中核反冲本底的主要来源。

利用$\mu$子持续产生电离的特点,可以在探测器外围部署$\mu$子反符合探测器,
当$\mu$子同时在反符合探测器和主探测器产生信号且符合一定条件时时,通过硬件触发或软件筛选,
可以将主探测器中事件标记为$\mu$子事件以区分信号的本底。但物理信号如CE$\nu$NS也有一定几率发生在$\mu$子穿越探测器时,
这时对$\mu$子事件的标记会使物理信号的探测效率降低,等效为曝光量(exposure)的损失,这种效应必须考虑到统计推断中,
否则将高估探测器中产生的信号。

地面附近的$\mu$子分布可以用Gaisser公式或Shukla公式描述。T.K.Gaisser在忽略地球曲率的情况下,
给出了$\mu$子通量随$\mu$子能量$E_\mu$和天顶角(zenith angle)$\theta$的分布。天顶角为入射粒子与地面法线间的夹角\cite{gaisser_cosmic_2016}。

\begin{align}
    \label{eq:gaisser}
    \frac{\mathrm{d}N_\mu}{\mathrm{d}E_\mu\mathrm{d}\Omega} &\approx 
    1400E_\mu^{-2.7}/\left(\si{m^2\cdot s\cdot GeV\cdot sr}\right)\left(\frac{1}{1+\frac{1.1E_\mu\cos\theta}{\epsilon_\pi}}+\frac{0.054}{1+\frac{1.1E_\mu\cos\theta}{\epsilon_\kappa}}\right)
\end{align}

其中$\epsilon_\pi\approx115\si{GeV},\epsilon_\kappa\approx850\si{GeV}$。
Gaisser公式只在高能区间($E_\mu>(100/\cos\theta)\si{GeV}$)适用。

P.Shukla等人在考虑地球曲率的条件下给出了类似的分布\cite{shukla_energy_2018},如式\ref{eq:shukla}:

\begin{align}
    \label{eq:shukla}
    I\left(E_\mu,\theta\right) &= I_0 N\left(E_0+E_\nu\right)^{-n}\left(1 + \frac{E_\mu}{\epsilon_\mu}\right)^{-1}D(\theta)^{-(n-1)} \\
    D(\theta) &= \sqrt{\frac{R^2}{d^2}\cos^2\theta+2\frac{R}{d}+1}-\frac{R}{d}\cos\theta
\end{align}

其中$I_0$为绝对流强,$N$为归一化参数,$n$为天顶角分布的幂次,$\epsilon_\mu,\frac{R}{d}$为经验公式中的自由参数。
使用不同地点测量得到的$\mu$子分布可以拟合得到不同结果,这里我们使用Shukla通过日本筑波市附近的测量数据得到的拟合结果,具体参数见表\ref{tab:shukla}。
Shukla公式在低能区域和高能区域都与数据符合得较好。

\begin{table}
  \centering
  \caption{描述$\mu$子分布的Shukla公式采用的参数列表}
  \begin{tabular}{cc}
    \toprule
    符号 & 取值 \\
    \midrule
    $I_0$ & $70.7\si{m^{-2}\cdot s^{-1}\cdot sr^{-1}}$ \\
    $n$ & 3.01 \\
    $E_0$ & $4.19\si{GeV}$ \\
    $\epsilon_\mu$ & $854\si{GeV}$ \\
    $R/d$ & $174$ \\
    \bottomrule
  \end{tabular}
  \label{tab:shukla}
\end{table}

式\ref{eq:shukla}中天顶角$\theta$相关的项可以从$E_\mu$中解耦合,$\theta$和$E_\mu$的分布如图\ref{fig:shukla_distribution}。

\begin{figure}
    \centering
    \includesvg[width=1.0\linewidth]{figures/shukla_distribution.svg}
    \caption{\label{fig:shukla_distribution} Shukla公式中$\theta$和$E_\mu$的分布,$\theta$主要集中在夹角较小的区域,
    因为天顶角更大的$\mu$子与大气层中物质的相互作用路程更长,损失更多;$E_\mu$主要集中在低能部分。}
\end{figure}

到达地面附近的$\mu$子包含了$\mu^-$和$\mu^+$。
CMS于2010年测量得到了地表附近的$\mu$子电荷比(charge ratio, $I_{\mu^+}/I_{\mu^-}$)约为$1.2766\pm0.0045$\cite{the_cms_collaboration_measurement_2010}。
$\mu$子电荷比在$\mu$子动量小于$100\si{GeV/c}$时与能量几乎无关,且在更高动量的区域略增大。
考虑到第\ref{sec:muon_nr}节中讨论的$\mu$子核反冲本底主要由较低能量的$\mu^-$贡献,更高的电荷比意味着更低的核反冲本底,本文保守地使用不随能量变化的电荷比。

模拟中将从屏蔽体外部一个足够的水平平面上均匀取点作为$\mu$子的入射位置,通过Shukla公式对$\mu$子的入射角和能量进行采样,
模拟几何如图\ref{fig:muon_inject}。

\begin{figure}
  \centering
  \includegraphics[width=0.6\linewidth]{figures/muon_inject.pdf}
  \caption{\label{fig:muon_inject} $\mu$子入射屏蔽体和探测器的示意图。
  最顶部平面为$\mu$子初始位置,蓝色圆柱体为液氙探测器。}
\end{figure}

Geant4模拟中设置中心祝探测器和液氙反符合屏蔽层为灵敏探测器,并记录两种灵敏探测器中的能量沉积。
因液氙中相近时间且相近空间位置的能量沉积并不能被有效区分,
得到模拟结果后将使用聚类算法DBSCAN(Density-Based Spatial Clustering of Applications with Noise)\cite{ester_density-based_1996,schubert_dbscan_2017}合并临近时空中的能量沉积。

\subsection{核反冲本底}
\label{sec:muon_nr}

$\mu$子主要通过两种方式产生核反冲。高能$\mu$子可以直接与原子核发生库伦散射使原子核获得动能;
低能$\mu$子有跟高的概率被原子核俘获,之后放出一个或几个中子。

\subsection{电子反冲本底}

\section{材料本底}

\subsection{光电倍增管本底}
\label{sec:pmt_background}

\subsection{屏蔽体材料本底}

\subsection{液氙中惰性元素本底}
