% !TeX root = ../main.tex

\chapter{引言}

\section{研究背景与方法}

中微子是一种标准模型中的轻子,主要参与弱相互作用,有三种``味''。
上世纪三十年代,为了解决$\beta$衰变中的能量守恒问题,中微子作为一种假想粒子被引入。
因为中微子与普通物质(原子核与电子)的作用截面极小以致于极难探测,
直到1956年,借助反应堆中微子的超高流强和精巧的探测器设计,
C.L.Cowan 与 F.Reines 才探测到反应堆放出的电子反中微子(electron antineutrino)与质子之间的逆$\beta$衰变(inverse beta decay, IBD)反应\cite{cowan_detection_1956}。
在标准模型中,中微子被预测为一种不带电的无质量费米子,但是在二十一世纪初,
超级神冈实验(Super-Kamiokande)\cite{collaboration_evidence_1998}与萨德伯里中微子观测实验(SNO)\cite{sno_collaboration_direct_2002}
对大气中微子和太阳中微子的观测结果表明,中微子会在三种``味''之间发生转换,这同时意味着中微子之间的质量差异。
这是中微子展现出超出标准模型预言性质的一个证据。

中微子与原子核的相干弹性散射(Coherent Elastic Neutrino-Nucleus Scattering, CE$\nu$NS)是一种中微子与原子核的相互作用。
当中微子通过中性流(neutral-current)与原子核散射时,会与原子核之间以 Z 玻色子(Z boson)交换动能和动量。
当交换的动量足够低时,Z 玻色子将视原子核为一个整体进行相互作用,结果是原子核获得了一个很小的动能。
但中微子与原子核整体反应导致反应截面急剧升高,使得CE$\nu$NS成为$\si{MeV}$中微子与物质的主要相互作用,也为我们提供了测量该极低能过程的机会。
作为一种典型的弱相互作用,CE$\nu$NS测量可以限制某些弱相互作用相关的参数,同时为超标准模型物理提供契机。
同时CE$\nu$NS是下一代暗物质直接测量实验的重要本底\cite{ohare_fog_2021},对CE$\nu$NS的研究也会辅助提升未来的暗物质实验精度。

2017年,COHERENT 合作组在美国橡树岭国家实验室的散裂中子源(The Spallation Neutron Source (SNS) at Oak Ridge National Laboratory)
上通过碘化铯($\mathrm{CsI}$)探测器首次以高置信度探测到了CE$\nu$NS过程。
我们计划设计建造一台30千克量级的液氙时间投影室(liquid xenon time projection chamber, LXeTPC)
或一台5千克的高纯锗(high purity Germanium, HPGe)探测器对反应堆电子反中微子进行探测。
相比于散裂中子源上中微子,反应堆中微子能量更低(不超过 10$\si{MeV}$),对其的测量是现有实验的良好补充。

在探测器的研制设计中,对探测器关键本底的研究和预测本底对实验灵敏度的影响至关重要。
不同于XENON、LZ等地下实验,地面运行的LXeTPC不具有山体屏蔽,宇宙线$\mu$子流强将上升几个数量级。
对$\mu$子穿过探测器的过程中产生额外本底的控制,是实验成功的关键因素之一。同时探测器材料中产生的本底也不可忽略。
我将建立一套本底模拟框架和一套灵敏度预测框架,重点研究$\mu$致本底和材料本底对实验的影响,并最终给出实验的灵敏度预测。

\section{论文结构与章节安排}

本文各章节组织如下:

第一章为引言。简单介绍研究背景和方法。

第二章为CE$\nu$NS的研究背景。主要介绍CE$\nu$NS的发生过程和现有实验对其的探测方法和进展。

第三章介绍探测器工作原理。从探测器工作原理和探测器特性的角度,介绍LXeTPC与HPGe探测器对物理信号的探测过程,这一部分为接下来的LXeTPC和HPGe上的信号模型研究提供支撑。

第四章创建信号模型。基于探测器的工作原理和以往对反应堆中微子的研究,创建两种探测器上的CE$\nu$NS信号模型。信号模型将作为灵敏度预测的关键输入。

第五章对关键本底进行模拟。建立LXeTPC实验本底的模拟框架,研究$\mu$致本底和材料本底的水平压低方法,研究结果将作为本底模型输入灵敏度预测框架。

第六章进行灵敏度预测。基于第四、五章的分析,建立灵敏度预测框架,以中微子超标准模型有效相互作用、低动量转移下的弱混合角和低能核反冲的光电产额为例预测探测器在模拟得到的本底水平下,对关键物理目标的探测能力。

第七章为总结和展望。本章总结测量CE$\nu$NS的灵敏度预测工作思路,并提出当前工作的局限性,为未来的相关工作提出展望。
